This is a short and rather shallow guide to the codebase. Starting from main.\+cpp, you\textquotesingle{}ll see the crucial steps and data structures involved in solving the terrace detection and enumeration problem.

\subsubsection*{0) Interpreting the C\+LI Arguments}

If two file paths are provided, it is assumed that the first path points to the tree file (.nwk) and the second path points to the data file (.data).

If one file path Y/X is provided, it is assumed that the files X.\+nwk and X.\+data are present in the directory Y.

\subsubsection*{1) Parsing the Input Files}

This is done by calling \hyperlink{namespaceterraces_add61915a31828774ee0371d443031c29}{terraces\+::parse\+\_\+nwk} and \hyperlink{namespaceterraces_af52559863b67502f00d68853f50c69af}{terraces\+::parse\+\_\+bitmatrix}. Both throw exceptions, if the input files are not in the right format. If the .data file does not contain a species that possesses all gene sites, this is denoted by a terraces\+::none value in the std\+::pair returned by \hyperlink{namespaceterraces_af52559863b67502f00d68853f50c69af}{terraces\+::parse\+\_\+bitmatrix}. In this case, the current course of action is to exit with error code 1.

\subsubsection*{2) Rerooting the Input Tree}